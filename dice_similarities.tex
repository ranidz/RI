\documentclass{article}
\usepackage[utf8]{inputenc}
\usepackage[T1]{fontenc}
\usepackage{amsmath}
\usepackage{booktabs}
\usepackage{array}
\usepackage[margin=1in]{geometry}
\begin{document}

\title{Document Similarity Calculation Using Dice Coefficient}
\author{Term Weight Calculator}
\maketitle

\section{Formula Used}
The Dice coefficient is used to calculate the similarity between a query $q$ and a document $d_j$:

\begin{equation}
Dice(q, d_j) = \frac{2 \times \sum_{i=1}^{n} w_{iq} \times w_{ij}}{\sum_{i=1}^{n} (w_{iq})^2 + \sum_{i=1}^{n} (w_{ij})^2}
\end{equation}

Where:
\begin{itemize}
\item $w_{iq}$ is the weight of term $i$ in the query $q$
\item $w_{ij}$ is the weight of term $i$ in document $d_j$
\item $n$ is the total number of terms
\end{itemize}

The Dice coefficient produces a value between 0 and 1, where 1 indicates perfect similarity and 0 indicates no similarity.

\section{Term Weights}
The term weights used for calculation are:

\begin{table}[h]
\centering
\begin{tabular}{lcccc}
\toprule
Term & D1 & D2 & D3 & Q \\
\midrule
algorithme & 0.000 & 0.000 & 1.000 & 0.000 \\
base & 0.000 & 1.000 & 0.000 & 0.000 \\
c++ & 0.000 & 1.000 & 0.000 & 0.000 \\
java & 0.000 & 1.000 & 0.000 & 2.000 \\
langage & 1.000 & 1.000 & 1.000 & 1.000 \\
programmation & 1.322 & 0.000 & 0.661 & 0.000 \\
python & 2.000 & 0.000 & 0.000 & 2.000 \\
programme & 0.000 & 0.000 & 1.000 & 0.000 \\
traitement & 2.000 & 0.000 & 0.000 & 0.000 \\
texte & 2.000 & 0.000 & 0.000 & 0.000 \\
traduire & 0.000 & 0.000 & 1.000 & 0.000 \\
utilis\'{e} & 1.322 & 0.000 & 0.661 & 0.000 \\
\bottomrule
\end{tabular}
\caption{Term weights for documents and query}
\end{table}

\section{Dice Coefficient Calculations}

\subsection{Dice Coefficient between Query and Document D1}
\begin{align}
Dice(q, D1) &= \frac{2 \times \sum_{i=1}^{n} w_{iq} \times w_{i,D1}}{\sum_{i=1}^{n} (w_{iq})^2 + \sum_{i=1}^{n} (w_{i,D1})^2} \\
\end{align}

First, let's calculate the dot product:
\begin{align}
\sum_{i=1}^{n} w_{iq} \times w_{i,D1} &= 0.000 \times 0.000 + 0.000 \times 0.000 + 0.000 \times 0.000 + 0.602 \times 0.000 + \\
&\quad 0.301 \times 0.301 + 0.000 \times 0.398 + 0.602 \times 0.602 + 0.000 \times 0.000 + \\
&\quad 0.000 \times 0.602 + 0.000 \times 0.602 + 0.000 \times 0.000 + 0.000 \times 0.398 \\
&= 0 + 0 + 0 + 0 + 0.091 + 0 + 0.362 + 0 + 0 + 0 + 0 + 0 \\
&= 0.453
\end{align}

Next, let's calculate the sum of squares for the query:
\begin{align}
\sum_{i=1}^{n} (w_{iq})^2 &= (0.000)^2 + (0.000)^2 + (0.000)^2 + (0.602)^2 + (0.301)^2 + (0.000)^2 + \\
&\quad (0.602)^2 + (0.000)^2 + (0.000)^2 + (0.000)^2 + (0.000)^2 + (0.000)^2 \\
&= 0 + 0 + 0 + 0.362 + 0.091 + 0 + 0.362 + 0 + 0 + 0 + 0 + 0 \\
&= 0.815
\end{align}

And the sum of squares for document D1:
\begin{align}
\sum_{i=1}^{n} (w_{i,D1})^2 &= (0.000)^2 + (0.000)^2 + (0.000)^2 + (0.000)^2 + (0.301)^2 + (0.398)^2 + \\
&\quad (0.602)^2 + (0.000)^2 + (0.602)^2 + (0.602)^2 + (0.000)^2 + (0.398)^2 \\
&= 0 + 0 + 0 + 0 + 0.091 + 0.158 + 0.362 + 0 + 0.362 + 0.362 + 0 + 0.158 \\
&= 1.493
\end{align}

Now we can calculate the Dice coefficient:
\begin{align}
Dice(q, D1) &= \frac{2 \times 0.453}{0.815 + 1.493} \\
&= \frac{0.906}{2.308} \\
&= 0.393
\end{align}

\subsection{Dice Coefficient between Query and Document D2}
\begin{align}
Dice(q, D2) &= \frac{2 \times \sum_{i=1}^{n} w_{iq} \times w_{i,D2}}{\sum_{i=1}^{n} (w_{iq})^2 + \sum_{i=1}^{n} (w_{i,D2})^2} \\
\end{align}

First, let's calculate the dot product:
\begin{align}
\sum_{i=1}^{n} w_{iq} \times w_{i,D2} &= 0.000 \times 0.000 + 0.000 \times 0.301 + 0.000 \times 0.301 + 0.602 \times 0.301 + \\
&\quad 0.301 \times 0.301 + 0.000 \times 0.000 + 0.602 \times 0.000 + 0.000 \times 0.000 + \\
&\quad 0.000 \times 0.000 + 0.000 \times 0.000 + 0.000 \times 0.000 + 0.000 \times 0.000 \\
&= 0 + 0 + 0 + 0.181 + 0.091 + 0 + 0 + 0 + 0 + 0 + 0 + 0 \\
&= 0.272
\end{align}

We already calculated the sum of squares for the query: 0.815

Now, let's calculate the sum of squares for document D2:
\begin{align}
\sum_{i=1}^{n} (w_{i,D2})^2 &= (0.000)^2 + (0.301)^2 + (0.301)^2 + (0.301)^2 + (0.301)^2 + (0.000)^2 + \\
&\quad (0.000)^2 + (0.000)^2 + (0.000)^2 + (0.000)^2 + (0.000)^2 + (0.000)^2 \\
&= 0 + 0.091 + 0.091 + 0.091 + 0.091 + 0 + 0 + 0 + 0 + 0 + 0 + 0 \\
&= 0.364
\end{align}

Now we can calculate the Dice coefficient:
\begin{align}
Dice(q, D2) &= \frac{2 \times 0.272}{0.815 + 0.364} \\
&= \frac{0.544}{1.179} \\
&= 0.461
\end{align}

\subsection{Dice Coefficient between Query and Document D3}
\begin{align}
Dice(q, D3) &= \frac{2 \times \sum_{i=1}^{n} w_{iq} \times w_{i,D3}}{\sum_{i=1}^{n} (w_{iq})^2 + \sum_{i=1}^{n} (w_{i,D3})^2} \\
\end{align}

First, let's calculate the dot product:
\begin{align}
\sum_{i=1}^{n} w_{iq} \times w_{i,D3} &= 0.000 \times 0.301 + 0.000 \times 0.000 + 0.000 \times 0.000 + 0.602 \times 0.000 + \\
&\quad 0.301 \times 0.301 + 0.000 \times 0.199 + 0.602 \times 0.000 + 0.000 \times 0.301 + \\
&\quad 0.000 \times 0.000 + 0.000 \times 0.000 + 0.000 \times 0.301 + 0.000 \times 0.199 \\
&= 0 + 0 + 0 + 0 + 0.091 + 0 + 0 + 0 + 0 + 0 + 0 + 0 \\
&= 0.091
\end{align}

We already calculated the sum of squares for the query: 0.815

Now, let's calculate the sum of squares for document D3:
\begin{align}
\sum_{i=1}^{n} (w_{i,D3})^2 &= (0.301)^2 + (0.000)^2 + (0.000)^2 + (0.000)^2 + (0.301)^2 + (0.199)^2 + \\
&\quad (0.000)^2 + (0.301)^2 + (0.000)^2 + (0.000)^2 + (0.301)^2 + (0.199)^2 \\
&= 0.091 + 0 + 0 + 0 + 0.091 + 0.040 + 0 + 0.091 + 0 + 0 + 0.091 + 0.040 \\
&= 0.444
\end{align}

Now we can calculate the Dice coefficient:
\begin{align}
Dice(q, D3) &= \frac{2 \times 0.091}{0.815 + 0.444} \\
&= \frac{0.182}{1.259} \\
&= 0.145
\end{align}

\section{Document Ranking}
Based on the Dice coefficient scores, the documents can be ranked in order of relevance to query Q:
\begin{enumerate}
\item D2 (Dice coefficient = 0.461)
\item D1 (Dice coefficient = 0.393)
\item D3 (Dice coefficient = 0.145)
\end{enumerate}

\section{Conclusion}
Using the Dice coefficient, document D2 has the highest similarity score with the query, making it the most relevant document. This differs from the RSV ranking, where D1 was ranked first. The Dice coefficient normalizes for document length, which can lead to different rankings compared to the raw dot product method.

\end{document}